\subsection*{Scientific Background: The Biological Imperative for Tissue-Specific Interactomics}

Protein-protein interactions (PPIs) orchestrate virtually all cellular processes, yet the \textbf{biological reality of PPIs is fundamentally context-dependent}. Within any specific tissue, only a subset of possible interactions can physically occur due to differential protein expression, abundance, and subcellular localization. This distinction between \textbf{possible} and \textbf{physiologically relevant} interactions has profound implications for understanding human disease.

Many complex diseases manifest in a tissue-specific manner despite involving ubiquitously expressed genes. Schizophrenia affects the brain; cardiomyopathies affect the heart; metabolic disorders affect the liver. Understanding why the same genetic variants produce tissue-selective phenotypes requires knowledge of which protein interactions actually occur in each tissue context. Current global interactome databases---while cataloging over 500,000 potential human PPIs---provide limited insight into \textbf{when and where} these interactions occur physiologically.

This gap severely constrains translational applications:
\begin{itemize}
    \item \textbf{Disease mechanism:} Without tissue context, it is difficult to distinguish causal interactions from bystanders.
    \item \textbf{Drug targeting:} Inhibiting a PPI active in healthy tissues may cause off-target toxicity.
    \item \textbf{Biomarker discovery:} Tissue-specific interaction disruptions may serve as disease signatures.
\end{itemize}

