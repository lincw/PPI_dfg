\documentclass{scrartcl}

\usepackage[utf8]{inputenc}
\usepackage[english]{proposal}

\setboolean{finalcompile}{true}

\input{Header.tex}

\ihead*{DFG-form 53.01 - 09/25}

\linespread{1.2}
\setlength{\parskip}{0.3em}

\usepackage{enumitem}
\setlist{noitemsep, topsep=0pt, parsep=0pt, partopsep=0pt}

% Define colors for Gantt chart
\definecolor{barblue}{RGB}{51, 102, 153}
\definecolor{groupblue}{RGB}{51, 102, 153}
\definecolor{linkred}{RGB}{160, 0, 0}

\RedeclareSectionCommand[
  beforeskip=-1ex plus -1ex minus -.2ex,
  afterskip=0.3ex plus .2ex
]{section}
\RedeclareSectionCommand[
  beforeskip=-1ex plus -1ex minus -.2ex,
  afterskip=0.2ex plus .2ex
]{subsection}
\RedeclareSectionCommand[
  beforeskip=-1ex plus -1ex minus -.2ex,
  afterskip=0.2ex plus .2ex
]{subsubsection}

\crefname{figure}{Figure}{Figures}

%%%%%%%%%%%%%%%%%%%%%%%%%%%%%%%%%%%%%%%%%%%%%%%%%%%%%%%%%%%%%%%%%%%%%%%%%%%%% 
%%%%  TITLE PAGE  %%%%%%%%%%%%%%%%%%%%%%%%%%%%%%%%%%%%%%%%%%%%%%%%%%%%%%%%%%% 
%%%%%%%%%%%%%%%%%%%%%%%%%%%%%%%%%%%%%%%%%%%%%%%%%%%%%%%%%%%%%%%%%%%%%%%%%%%%% 

\newcommand{\applicants}{Dr. Chung-Wen Lin}
\newcommand{\project}{LLM-Enhanced Tissue-Specific Protein-Protein Interaction Probability Estimation}

\addtocategory{important}{
fuchs2025varvamp,lamkiewicz2024ribap,galeone2025decoding,
}

\begin{document}
\pagestyle{empty}
\setcounter{page}{1}

\section*{Abstract}

Protein-protein interactions (PPIs) are fundamental to cellular function but vary dramatically across tissues. Current computational methods, relying largely on gene co-expression, fail to capture this context-specificity, creating a critical bottleneck for precision medicine. This proposal introduces a novel \textbf{Hybrid Neuro-Symbolic framework} that leverages Large Language Models (LLMs) as biological reasoning engines to predict tissue-specific interaction probabilities.

By integrating multi-modal data---including RNA expression, protein abundance (PaxDB), and subcellular localization---into structured prompts, our approach moves beyond statistical correlation to semantic reasoning. Preliminary validation on 205 schizophrenia-related brain interactions demonstrates a \textbf{31.9\% improvement (F1=0.727)} over baselines. The project will:
\begin{enumerate}
    \item Generate a genome-scale tissue-resolved interactome using a high-throughput inference pipeline on a dedicated Mac Studio workstation;
    \item Validate predictions through a tiered structural strategy (ESMFold/AlphaFold-Multimer) and heterogeneous Graph Neural Networks (GNNs);
    \item Deploy automated literature-mining agents to maintain a "living" knowledge base.
\end{enumerate}
This work will deliver the first comprehensive, structurally-validated atlas of the human tissue interactome, directly empowering disease mechanism discovery and drug target prioritization.



\cleardoublepage
\pagestyle{plain}

%%%%%%%%%%%%%%%%%%%%%%%%%%%%%%%%%%%%%%%%%%%%%%%%%%%%%%%%%%%%%%%%%%%%%%%%%%%%%%% 
%%%%  PROJECT DESCRIPTION - PROJECT PROPOSALS  %%%%%%%%%%%%%%%%%%%%%%%%%%%%%%%% 
%%%%%%%%%%%%%%%%%%%%%%%%%%%%%%%%%%%%%%%%%%%%%%%%%%%%%%%%%%%%%%%%%%%%%%%%%%%%%%% 
{\raggedright{} \normalsize \bfseries 
	Project Description -- Project Proposals \newline \par
	\applicants{} \newline \par
	\project{} \par
	\rule{\textwidth}{0.5pt} \par
}

%%%%%%%%%%%%%%%%%%%%%%%%%%%%%%%%%%%%%%%%%%%%%%%%%%%%%%%%%%%%%%%%%%%%%%%%%%%%% 
%%%%  STATE OF THE ART AND PRELIMINARY WORK %%%%%%%%%%%%%%%%%%%%%%%%%%%%%%%%% 
%%%%%%%%%%%%%%%%%%%%%%%%%%%%%%%%%%%%%%%%%%%%%%%%%%%%%%%%%%%%%%%%%%%%%%%%%%%%% 
\section{Starting Point}
\label{sec:work-report}

\subsection*{Scientific Background: The Biological Imperative for Tissue-Specific Interactomics}

Protein-protein interactions (PPIs) orchestrate virtually all cellular processes, yet the \textbf{biological reality of PPIs is fundamentally context-dependent}. Within any specific tissue, only a subset of possible interactions can physically occur due to differential protein expression, abundance, and subcellular localization. This distinction between \textbf{possible} and \textbf{physiologically relevant} interactions has profound implications for understanding human disease.

Many complex diseases manifest in a tissue-specific manner despite involving ubiquitously expressed genes. Schizophrenia affects the brain; cardiomyopathies affect the heart; metabolic disorders affect the liver. Understanding why the same genetic variants produce tissue-selective phenotypes requires knowledge of which protein interactions actually occur in each tissue context. Current global interactome databases---while cataloging over 500,000 potential human PPIs---provide limited insight into \textbf{when and where} these interactions occur physiologically.

This gap severely constrains translational applications:
\begin{itemize}
    \item \textbf{Disease mechanism:} Without tissue context, it is difficult to distinguish causal interactions from bystanders.
    \item \textbf{Drug targeting:} Inhibiting a PPI active in healthy tissues may cause off-target toxicity.
    \item \textbf{Biomarker discovery:} Tissue-specific interaction disruptions may serve as disease signatures.
\end{itemize}



\subsection*{State of the art: From Statistical Correlation to Biological Reasoning}

The reconstruction of tissue-specific interactomes represents a frontier in systems biology, moving the field from static "parts lists" to context-aware functional models. However, current computational methodologies face a fundamental "reasoning gap" that limits their predictive accuracy and biological interpretability.

\paragraph{Limitations of Co-expression and Statistical Correlation}
Traditionally, tissue-specific protein-protein interactions (PPIs) have been inferred by filtering global interactomes through mRNA co-expression data. While co-expression is a necessary prerequisite for interaction (the law of mass action), it is demonstrably insufficient. A significant fraction of co-expressed proteins never interact due to mismatched subcellular localization, incompatible molecular functions, or low protein-level abundance. Recent studies, notably by \textcite{LamanTrip2025}, have shown that protein co-abundance (derived from mass spectrometry) provides a superior signal compared to mRNA (AUROC 0.80 vs 0.70). However, even co-abundance remains a \emph{proxy} for interaction, failing to capture the structural and biochemical logic that dictates whether two proteins \emph{actually} bind in a specific physiological context.

\paragraph{Critique of Contemporary AI Architectures}
Modern deep learning approaches have attempted to address these limitations but remain constrained by their architectural assumptions:
\begin{itemize}
    \item \textbf{SPIDER} \cite{SPIDER} utilizes graph attention networks to predict cell-type-specific networks. While effective, it requires high-quality experimentally-derived networks for supervision, which are unavailable for the vast majority of human tissues.
    \item \textbf{PINNACLE} \cite{PINNACLE} generates context-aware representations across 156 cell types. However, its primary output consists of latent embeddings rather than explicit, probabilistic assessments of interaction occurrence. These "black-box" models lack a mechanism to integrate multi-modal semantic evidence (e.g., functional annotations and literature-derived context) in a biologically nuanced way.
\end{itemize}

\paragraph{The Emerging Paradigm: LLMs as Biological Reasoning Engines}
The recent breakthrough in Large Language Models (LLMs) offers a novel path forward. Unlike traditional machine learning models that treat biological data as purely numerical vectors, LLMs can process information semantically. As demonstrated in our preliminary work, LLMs can act as "biological reasoning engines," integrating diverse modalities---RNA expression, protein abundance, subcellular compartments, and functional logic---into a unified decision-making framework. This approach mimics the cognitive process of a systems biologist, evaluating not just \emph{if} two proteins are present, but \emph{why} they are likely to interact given the specific tissue environment.

By combining the **semantic reasoning** of LLMs with the **topological efficiency** of GNNs and the **physical rigor** of AlphaFold, this project aims to bridge the gap between statistical correlation and functional reality, creating the first truly reasoned, structurally-validated tissue-resolved human interactome.


% Preliminary Results
% This file is included in dfg.tex using % Preliminary Results
% This file is included in dfg.tex using % Preliminary Results
% This file is included in dfg.tex using \input{preliminary_results}

\subsubsection*{Preliminary results: LLM-driven rescue and enhancement of transcriptomic GNNs}

Current context-specific protein-protein interaction (PPI) estimation methods primarily rely on gene co-expression as a proxy for interaction. While effective for obligate complexes, these methods suffer from systematic limitations:

\begin{enumerate}
    \item Constitutive expression patterns can confound PPI tissue-specificity. Proteins expressed ubiquitously may appear to interact across all tissues regardless of functional context.
    \item Asymmetric abundance between interacting partners can bias tissue-specific inference. When one partner is highly abundant while the other is rate-limiting, expression-based methods may overestimate interaction occurrence.
    \item Post-translational regulation is often ignored.
\end{enumerate}

Here, I hypothesized that large language models (LLMs), by leveraging the vast biomedical literature, could rescue functional interactions missed by state-of-the-art expression-based graph neural networks (GNNs) and enhance GNN performance when integrated into a hybrid neuro-symbolic model.

I established a robust adversarial comparison across three approaches: \textit{baseline}, \textit{competitor}, and \textit{synthesis}. The \textit{competitor} is the proposed LLM-based probability framework. The \textit{baseline} is a self-supervised transcriptomic GNN (GATv2) optimized for 500 epochs per tissue. The \textit{synthesis} is a hybrid model combining topological (GNN) and semantic (LLM) signals. The evaluation covered 49 human tissues using a head-to-head subset of $\sim$1\,200 PPI-tissue pairs per tissue, validated against the independent Reactome pathway database \cite{Reactome}.

\paragraph*{Key findings}

\begin{enumerate}
    \item \textbf{The \textit{Rescue} phenomenon:} 
    Across 49 tissues, we identified 1\,935 specific interactions where the LLM-based framework strongly disagreed with the expression data. These \enquote{rescue} cases represent functional biology invisible to transcriptomics.

    \begin{table}[h]
    \centering
    \small
    \begin{tabular}{llc}
    \hline
    \textbf{Prediction group} & \textbf{Description} & \textbf{Pathway validation Rate} \\
    \hline
    \textbf{LLM rescue} & \textbf{Low GNN / High LLM} & \textbf{10.7\%} (1\,935 cases) \\
GNN unique & High GNN / Low LLM & 2.5\% (863 cases) \\
Random baseline & randomly sampled PPIs & $\sim$0.0\% \\
    \hline
    \end{tabular}
    \caption{Validation of PPI predictions against Reactome pathways. The LLM rescue group shows a significantly higher validation rate compared to GNN-unique predictions.}
    \label{tab:llm_rescue}
    \end{table}

    The LLM outputs showed superior pathway validation in 44 out of 49 tissues. Notably, in the low expression regime (bottom 50\%), the LLM outputs maintained a 16\% validation rate, compared to just 4\% for the GNN.

    \item \textbf{The \textit{Hybrid} advantage:} 
    To further evaluate whether combining methods yields superior results, a hybrid model (averaging GNN and LLM probabilities) was tested on 10 diverse tissues.
    The hybrid model achieved the highest area under the precision-recall curve in 7 out of 10 tissues.
    While the GNN captures strong co-expression signals (topological/abundance), the LLM captures functional context (semantic/regulatory).
    A key advantage is that the hybrid approach effectively filters false positives from the GNN and hallucinations from the LLM.

    \item \textbf{Semantic insight:} 
    Qualitative analysis reveals why the LLM framework succeeds. 
    For the rescued pair AGBL2--CEP70 (which showed high probability in the cerebral cortex), the LLM framework noted that: \enquote{both proteins act at centrosomal structures and may intersect in microtubule regulation... The interaction is relevant to neuronal ciliary and centrosomal contexts in cortex.}
    The LLM correctly weighted the functional context (centrosome localization) over transcriptional noise (low basal expression of AGBL2), effectively integrating prior knowledge absent from numerical expression matrices.
\end{enumerate}

\paragraph*{Conclusion}
These results demonstrate that the LLM captures functional relationships invisible to transcriptomics-based GNNs, a finding validated by higher pathway enrichment.
The hybrid neuro-symbolic approach outperforms either method individually, suggesting that the future of systems biology lies in integrating data-driven topology with literature-driven reasoning.



\subsubsection*{Preliminary results: LLM-driven rescue and enhancement of transcriptomic GNNs}

Current context-specific protein-protein interaction (PPI) estimation methods primarily rely on gene co-expression as a proxy for interaction. While effective for obligate complexes, these methods suffer from systematic limitations:

\begin{enumerate}
    \item Constitutive expression patterns can confound PPI tissue-specificity. Proteins expressed ubiquitously may appear to interact across all tissues regardless of functional context.
    \item Asymmetric abundance between interacting partners can bias tissue-specific inference. When one partner is highly abundant while the other is rate-limiting, expression-based methods may overestimate interaction occurrence.
    \item Post-translational regulation is often ignored.
\end{enumerate}

Here, I hypothesized that large language models (LLMs), by leveraging the vast biomedical literature, could rescue functional interactions missed by state-of-the-art expression-based graph neural networks (GNNs) and enhance GNN performance when integrated into a hybrid neuro-symbolic model.

I established a robust adversarial comparison across three approaches: \textit{baseline}, \textit{competitor}, and \textit{synthesis}. The \textit{competitor} is the proposed LLM-based probability framework. The \textit{baseline} is a self-supervised transcriptomic GNN (GATv2) optimized for 500 epochs per tissue. The \textit{synthesis} is a hybrid model combining topological (GNN) and semantic (LLM) signals. The evaluation covered 49 human tissues using a head-to-head subset of $\sim$1\,200 PPI-tissue pairs per tissue, validated against the independent Reactome pathway database \cite{Reactome}.

\paragraph*{Key findings}

\begin{enumerate}
    \item \textbf{The \textit{Rescue} phenomenon:} 
    Across 49 tissues, we identified 1\,935 specific interactions where the LLM-based framework strongly disagreed with the expression data. These \enquote{rescue} cases represent functional biology invisible to transcriptomics.

    \begin{table}[h]
    \centering
    \small
    \begin{tabular}{llc}
    \hline
    \textbf{Prediction group} & \textbf{Description} & \textbf{Pathway validation Rate} \\
    \hline
    \textbf{LLM rescue} & \textbf{Low GNN / High LLM} & \textbf{10.7\%} (1\,935 cases) \\
GNN unique & High GNN / Low LLM & 2.5\% (863 cases) \\
Random baseline & randomly sampled PPIs & $\sim$0.0\% \\
    \hline
    \end{tabular}
    \caption{Validation of PPI predictions against Reactome pathways. The LLM rescue group shows a significantly higher validation rate compared to GNN-unique predictions.}
    \label{tab:llm_rescue}
    \end{table}

    The LLM outputs showed superior pathway validation in 44 out of 49 tissues. Notably, in the low expression regime (bottom 50\%), the LLM outputs maintained a 16\% validation rate, compared to just 4\% for the GNN.

    \item \textbf{The \textit{Hybrid} advantage:} 
    To further evaluate whether combining methods yields superior results, a hybrid model (averaging GNN and LLM probabilities) was tested on 10 diverse tissues.
    The hybrid model achieved the highest area under the precision-recall curve in 7 out of 10 tissues.
    While the GNN captures strong co-expression signals (topological/abundance), the LLM captures functional context (semantic/regulatory).
    A key advantage is that the hybrid approach effectively filters false positives from the GNN and hallucinations from the LLM.

    \item \textbf{Semantic insight:} 
    Qualitative analysis reveals why the LLM framework succeeds. 
    For the rescued pair AGBL2--CEP70 (which showed high probability in the cerebral cortex), the LLM framework noted that: \enquote{both proteins act at centrosomal structures and may intersect in microtubule regulation... The interaction is relevant to neuronal ciliary and centrosomal contexts in cortex.}
    The LLM correctly weighted the functional context (centrosome localization) over transcriptional noise (low basal expression of AGBL2), effectively integrating prior knowledge absent from numerical expression matrices.
\end{enumerate}

\paragraph*{Conclusion}
These results demonstrate that the LLM captures functional relationships invisible to transcriptomics-based GNNs, a finding validated by higher pathway enrichment.
The hybrid neuro-symbolic approach outperforms either method individually, suggesting that the future of systems biology lies in integrating data-driven topology with literature-driven reasoning.



\subsubsection*{Preliminary results: LLM-driven rescue and enhancement of transcriptomic GNNs}

Current context-specific protein-protein interaction (PPI) estimation methods primarily rely on gene co-expression as a proxy for interaction. While effective for obligate complexes, these methods suffer from systematic limitations:

\begin{enumerate}
    \item Constitutive expression patterns can confound PPI tissue-specificity. Proteins expressed ubiquitously may appear to interact across all tissues regardless of functional context.
    \item Asymmetric abundance between interacting partners can bias tissue-specific inference. When one partner is highly abundant while the other is rate-limiting, expression-based methods may overestimate interaction occurrence.
    \item Post-translational regulation is often ignored.
\end{enumerate}

Here, I hypothesized that large language models (LLMs), by leveraging the vast biomedical literature, could rescue functional interactions missed by state-of-the-art expression-based graph neural networks (GNNs) and enhance GNN performance when integrated into a hybrid neuro-symbolic model.

I established a robust adversarial comparison across three approaches: \textit{baseline}, \textit{competitor}, and \textit{synthesis}. The \textit{competitor} is the proposed LLM-based probability framework. The \textit{baseline} is a self-supervised transcriptomic GNN (GATv2) optimized for 500 epochs per tissue. The \textit{synthesis} is a hybrid model combining topological (GNN) and semantic (LLM) signals. The evaluation covered 49 human tissues using a head-to-head subset of $\sim$1\,200 PPI-tissue pairs per tissue, validated against the independent Reactome pathway database \cite{Reactome}.

\paragraph*{Key findings}

\begin{enumerate}
    \item \textbf{The \textit{Rescue} phenomenon:} 
    Across 49 tissues, we identified 1\,935 specific interactions where the LLM-based framework strongly disagreed with the expression data. These \enquote{rescue} cases represent functional biology invisible to transcriptomics.

    \begin{table}[h]
    \centering
    \small
    \begin{tabular}{llc}
    \hline
    \textbf{Prediction group} & \textbf{Description} & \textbf{Pathway validation Rate} \\
    \hline
    \textbf{LLM rescue} & \textbf{Low GNN / High LLM} & \textbf{10.7\%} (1\,935 cases) \\
GNN unique & High GNN / Low LLM & 2.5\% (863 cases) \\
Random baseline & randomly sampled PPIs & $\sim$0.0\% \\
    \hline
    \end{tabular}
    \caption{Validation of PPI predictions against Reactome pathways. The LLM rescue group shows a significantly higher validation rate compared to GNN-unique predictions.}
    \label{tab:llm_rescue}
    \end{table}

    The LLM outputs showed superior pathway validation in 44 out of 49 tissues. Notably, in the low expression regime (bottom 50\%), the LLM outputs maintained a 16\% validation rate, compared to just 4\% for the GNN.

    \item \textbf{The \textit{Hybrid} advantage:} 
    To further evaluate whether combining methods yields superior results, a hybrid model (averaging GNN and LLM probabilities) was tested on 10 diverse tissues.
    The hybrid model achieved the highest area under the precision-recall curve in 7 out of 10 tissues.
    While the GNN captures strong co-expression signals (topological/abundance), the LLM captures functional context (semantic/regulatory).
    A key advantage is that the hybrid approach effectively filters false positives from the GNN and hallucinations from the LLM.

    \item \textbf{Semantic insight:} 
    Qualitative analysis reveals why the LLM framework succeeds. 
    For the rescued pair AGBL2--CEP70 (which showed high probability in the cerebral cortex), the LLM framework noted that: \enquote{both proteins act at centrosomal structures and may intersect in microtubule regulation... The interaction is relevant to neuronal ciliary and centrosomal contexts in cortex.}
    The LLM correctly weighted the functional context (centrosome localization) over transcriptional noise (low basal expression of AGBL2), effectively integrating prior knowledge absent from numerical expression matrices.
\end{enumerate}

\paragraph*{Conclusion}
These results demonstrate that the LLM captures functional relationships invisible to transcriptomics-based GNNs, a finding validated by higher pathway enrichment.
The hybrid neuro-symbolic approach outperforms either method individually, suggesting that the future of systems biology lies in integrating data-driven topology with literature-driven reasoning.



%%%%%%%%%%%%%%%%%%%%%%%%%%%%%%%%%%%%%%%%%%%%%%%%%%%%%%%%%%%%%%%%%%%%%%%%%%%%%
%%%%  OBJECTIVES AND WP %%%%%%%% %%%%%%%%%%%%%%%%%%%%%%%%%%%%%%%%%%%%%%%%%%%%
%%%%%%%%%%%%%%%%%%%%%%%%%%%%%%%%%%%%%%%%%%%%%%%%%%%%%%%%%%%%%%%%%%%%%%%%%%%%%
\section{Objectives and work program}

\subsection{Anticipated total duration of the project}
Financial support is requested for three years.

\subsection{Objectives}
\let\oldpara=\theparagraph
\addtocounter{secnumdepth}{1}
\renewcommand{\theparagraph}{Goal \arabic{paragraph}}

\paragraph{\textnormal{Generate a comprehensive tissue-specific PPI probability database.}}
We will produce tissue-specific probability scores for $\approx$170,000 known protein-protein interactions across 50 human tissues, creating a novel three-score framework (likelihood, confidence, concordance) that captures both biological probability and data quality.

\paragraph{\textnormal{Validate LLM-derived predictions through orthogonal evidence.}}
We will demonstrate that LLM-based biological reasoning captures functional relationships invisible to expression-based methods, validated through structural analysis (AlphaFold-Multimer) and independent pathway databases (Reactome).

\paragraph{\textnormal{Create a sustainable, community-accessible resource.}}
We will deploy an open web platform and a distilled AI model, hosted on dedicated hardware to ensure long-term availability beyond the funding period.

\subsection{Work program including proposed research methods}

The three work packages form an integrated pipeline (\cref{fig:timeline}): \textbf{WP1} generates tissue-specific probability scores using LLM-based biological reasoning and produces a distilled model for efficient inference. \textbf{WP2} validates predictions through structural analysis and extends coverage to the full proteome via a hybrid neuro-symbolic GNN. \textbf{WP3} creates a sustainable community resource with continuous literature-based updates. The requested Mac Studio workstation serves all three phases: as a high-throughput inference engine (WP1), model training platform (WP2), and long-term hosting server (WP3).

\hrulefill

\paragraph{WP1: Generation of a Tissue-Specific PPI Probability Profile via LLM Reasoning (Year 1)}
\label{wp:1}

\textbf{Objective:} To utilize the requested high-memory workstation, acquired at the project's commencement, as a biological reasoning engine to estimate the likelihood of $\approx$180,000 known protein-protein interactions (PPIs) occurring across 50 major human tissues, followed by knowledge distillation into efficient local models.

\subparagraph{Task 1.1: Multi-modal Data Integration and Prompt Engineering (Months 1-3)}
We will integrate five key biological data sources to construct context-aware prompts:
(1) consensus RNA expression (nTPM) from the Human Protein Atlas,
(2) protein abundance (ppm) from PaxDB v6.0,
(3) immunohistochemistry-based protein detection from the Human Protein Atlas,
(4) subcellular localization annotations, and
(5) tissue-specific Gaussian Mixture Model (GMM) expression profiles.
To ensure data integrity, we will implement a robust data harmonization pipeline using Ensembl IDs as the primary key, resolving identifier conflicts across multi-modal sources. The GMM approach automatically discovers biologically meaningful expression regimes (basal, housekeeping, tissue-relevant, tissue-specific) for each tissue, replacing arbitrary percentile cutoffs with data-driven, adaptive thresholds.

The requested Mac Studio will be used during this phase to perform initial pilot runs and prompt optimization using large-scale open-weights models (e.g., gpt-oss-120b or Llama-3-70b).
In preliminary studies on 205 experimentally validated brain-specific PPIs related to schizophrenia, our framework achieved an AUROC of 0.86, substantially outperforming expression-only baselines (AUROC 0.72).

\subparagraph{Task 1.2: High-Throughput "In Silico" Inference (Months 4-8)}
This task represents the core computational experiment utilizing the \textbf{Mac Studio (512GB RAM)}.
\begin{itemize}
    \item \textbf{Scale:} We will process the combinatorial space of $\approx$170,000 interactions $\times$ 50 tissues, resulting in $\approx$8.5 million unique queries.
    \item \textbf{Implementation:} We will deploy optimized large models locally on the 512GB unified memory architecture using mature, high-efficiency deployment tools for the Apple Silicon ecosystem (e.g., \textbf{Ollama}, MLX). Based on pilot experiments ($\approx$6 seconds per query), running 10 parallel model instances will complete all queries within approximately 8 weeks. To manage computational risk, the workflow supports dynamic load balancing, prioritizing interactions with high biological relevance or data completeness.
    \item \textbf{Three-Score Output:} For each PPI-tissue pair, the LLM generates three complementary scores: (1) \emph{Likelihood} (0--100): the biological probability of interaction occurrence; (2) \emph{Confidence} (0--100): the certainty of assessment based on data quality and completeness; and (3) \emph{Concordance}: a computationally derived measure of expression regime agreement between interaction partners, calculated from the GMM classifications. Importantly, the framework explicitly handles missing data (e.g., proteins lacking IHC or subcellular annotations) by treating absence as uncertainty rather than negative evidence, with appropriate penalties to the confidence score.
    \item \textbf{Hardware Lifecycle:} Following completion of the high-throughput inference phase, the Mac Studio will transition to support GNN training (WP2) and ultimately serve as a dedicated long-term hosting server for the public web platform (WP3), ensuring project sustainability beyond the funding period.
\end{itemize}

\subparagraph{Task 1.3: Knowledge Distillation and Efficient Model Creation (Months 9-12)}
The Mac Studio remains essential for the distillation phase.
\begin{itemize}
    \item \textbf{Dataset Construction:} Reasoning traces generated in Task 1.2 will be processed into instruction-following pairs, preserving the biological rationale alongside numerical scores.
    \item \textbf{Student Model Training:} We will perform supervised fine-tuning (SFT) of parameter-efficient models (e.g., Llama-3-8B) on the Mac Studio. The 512GB RAM allows for high-throughput training with large batch sizes even when using complex teacher-student distillation frameworks.
\end{itemize}

\noindent\myTip[1.0\textwidth]{t}{1}{\emph{\vspace*{0.1cm}\textbf{Summary of WP1}\newline
\textbf{Aim:} Generate a complete tissue-resolved interactome map and a portable prediction tool.\newline
\textbf{Method:} High-throughput inference using a novel three-score framework (likelihood, confidence, concordance) with GMM-based adaptive thresholds, followed by knowledge distillation.\newline
\textbf{Outcome:} A database of 8.5M tissue-specific probabilities and an open-source, lightweight AI model.
}}

\hrulefill

\paragraph{WP2: Structural Validation and Genome-Scale Expansion (Year 2)}
\label{wp:2}

\textbf{Objective:} To validate the biological plausibility of LLM-derived interactions using physical structural constraints and to extend predictions to the entire proteome via a Hybrid Neuro-Symbolic approach combining Graph Neural Networks (GNN) with LLM-derived semantic features.

\subparagraph{Task 2.1: Structural Validation via AlphaFold-Multimer (Months 13-18)}
To address the lack of high-throughput experimental validation, we will implement a Physical Gate by leveraging the institute's High-Performance Computing (HPC) resources.
\begin{itemize}
    \item \textbf{Target Selection:} We will select the top 5\% high-probability tissue-specific interactions identified in WP1 (approximately 400k pairs). Priority will be given to interfaces containing known pathogenic variants (from ClinVar) to maximize clinical relevance. Negative controls will be drawn from organ-specific PPIs that are definitively non-occurring in the target tissue.
    \item \textbf{Structural Prediction:} We will deploy AlphaFold-Multimer on the HPC cluster to predict the interface physical stability, specifically analyzing pLDDT and ipTM scores.
    \item \textbf{Dual-Gating Analysis:} We will correlate the Environmental Probability with the Physical Probability from HPC structural modeling.
\end{itemize}

\subparagraph{Task 2.2: Hybrid Neuro-Symbolic Extension via Graph Neural Networks (Months 19-24)}
While WP1 covers known interactions, we aim to discover novel interactions across the entire proteome. Preliminary results across 49 tissues demonstrate that LLM reasoning captures functional relationships invisible to expression-based methods: interactions ``rescued'' by the LLM (low expression signal, high LLM score) showed 4.3-fold higher pathway validation rates than expression-only predictions. Furthermore, a hybrid approach combining GNN topology with LLM semantics outperformed either method alone in 7 of 10 tested tissues.
\begin{itemize}
    \item \textbf{Heterogeneous Graph Construction:} We will construct tissue-specific Heterogeneous Graphs where nodes represent proteins and edges represent the LLM-derived probabilities. Using a heterogeneous architecture (e.g., Heterogeneous Graph Transformer) allows the model to explicitly account for distinct tissue-specific node features and edge types.
    \item \textbf{Training Strategy:} The GNN will be trained on the Student model's outputs to learn the general, transferable principles of tissue specificity. The Mac Studio transitions during this phase from high-throughput inference (WP1) to model development.
    \item \textbf{Novel Interaction Prediction:} The trained model will infer tissue-specific probabilities for protein pairs absent from HuRI and BioPlex, generating a complete predicted tissue-resolved interactome.
\end{itemize}

\noindent\myTip[1.0\textwidth]{t}{2}{\emph{\vspace*{0.1cm}\textbf{Summary of WP2}\newline
\textbf{Aim:} Validate predictions physically and extend them to the full genome.\newline
\textbf{Method:} AlphaFold structural prediction combined with a Hybrid Neuro-Symbolic GNN that integrates topological and LLM-derived semantic features.\newline
\textbf{Outcome:} A physically-validated, genome-scale atlas of tissue-specific protein interactions with orthogonal validation.}}

\hrulefill

\paragraph{WP3: Community Platform and Literature-Based Verification (Year 3)}
\label{wp:3}

\textbf{Objective:} To deploy automated AI agents for continuous literature-based verification, develop a user-friendly public resource, and demonstrate clinical utility through disease-specific case studies.

\subparagraph{Task 3.1: Automated Literature Mining Agents (Months 25-30)}
To scale validation beyond structural prediction, we will leverage the reasoning capabilities of our LLM infrastructure to mine the vast biomedical literature.
\begin{itemize}
    \item \textbf{Agent Development:} We will develop autonomous ``Research Agents'' powered by the distilled Student model from WP1. These agents will query PubMed/PMC for specific protein pairs, retrieving abstracts and full-text sections that mention both proteins in tissue-specific contexts.
    \item \textbf{Evidence Synthesis:} Agents will identify indirect evidence supporting or refuting predicted interactions, including: co-citation in tissue-specific studies, shared pathway annotations in KEGG/Reactome, co-occurrence in disease-gene databases (e.g., DisGeNET), and mentions of regulatory relationships. To avoid circular reasoning, we will exclude training-set literature and prioritize post-2024 publications.
    \item \textbf{Continuous Update:} The system will run periodically (monthly), updating confidence scores as new experimental evidence emerges, effectively creating a ``living'' interactome that evolves with the literature.
\end{itemize}

\subparagraph{Task 3.2: Web Platform and Disease Application (Months 31-36)}
The final phase focuses on dissemination and demonstrating translational value.
\begin{itemize}
    \item \textbf{Public Web Server:} We will launch a user-friendly web portal where users can query any protein pair and receive a comprehensive report containing: (1) Tissue-specific probabilities across 50 tissues with GMM-based expression context, (2) Pre-computed structural interface metrics (pLDDT, ipTM), and (3) Automated literature summaries with evidence provenance. The platform will be containerized (Docker/Singularity) for long-term maintainability and designed for potential integration with community resources (e.g., STRING, IntAct, ELIXIR).
    \item \textbf{Long-Term Sustainability:} The Mac Studio requested in this proposal will transition from its primary role as a high-throughput inference engine (WP1) to a dedicated hosting server post-funding. Its 512GB unified memory can simultaneously serve the web platform, run the distilled Student model for on-demand queries, and execute periodic literature-mining updates---ensuring the resource remains ``living'' without recurring cloud infrastructure costs. This hardware-as-legacy approach guarantees continued community access beyond the funding period.
    \item \textbf{Disease Case Studies:} We will demonstrate the platform's utility through two complementary case studies:
    \begin{enumerate}
        \item \emph{Schizophrenia (brain):} We select this disease because experimentally validated brain-specific PPIs (n=205) from recent synaptosome pulldown studies provide rigorous ground truth for benchmarking. The brain represents a unique challenge due to its high alternative splicing complexity.
        \item \emph{Cardiac arrhythmias (heart):} Ion channel complexes identified as hub proteins in our brain analysis have cardiac paralogs. The heart provides a distinct physiological context characterized by high metabolic demand and mechanical stress. Validating across these two extremes ensures the framework's generalizability across diverse organ systems.
    \end{enumerate}
\end{itemize}

\noindent\myTip[1.0\textwidth]{t}{3}{\emph{\vspace*{0.1cm}\textbf{Summary of WP3}\newline
\textbf{Aim:} Disseminate results and provide continuous, automated validation.\newline
\textbf{Method:} LLM-driven literature agents with circular-reasoning safeguards, containerized web platform, and dual disease case studies.\newline
\textbf{Outcome:} A ``living'', accessible tissue-specific interactome resource for the scientific community, hosted on dedicated hardware that ensures long-term sustainability beyond the funding period.
}}


\section{Innovation and Impact}

\subsection{Methodological Innovations}
\begin{enumerate}
    \item \textbf{LLM-Based Biological Reasoning Framework:} Novel application of large language models as biological reasoning engines, enabling integration of multi-modal evidence (expression, abundance, localization, function) that purely numerical approaches cannot achieve.
    \item \textbf{Three-Score Assessment System:} A new framework combining likelihood, confidence, and concordance scores that provides nuanced, probabilistic assessments rather than binary predictions.
    \item \textbf{Hybrid Neuro-Symbolic Architecture:} Integration of LLM semantic reasoning with GNN topological learning, capturing complementary biological signals as demonstrated by our preliminary results (4.3-fold improvement in pathway validation).
    \item \textbf{Knowledge Distillation Pipeline:} Application of teacher-student model architectures to biological prediction, creating efficient, deployable models that preserve the reasoning capabilities of larger systems.
\end{enumerate}

\subsection{Scientific Impact}
\begin{itemize}
    \item \textbf{Disease Research:} Tissue-specific pathway analysis for complex diseases with tissue-selective manifestation.
    \item \textbf{Drug Discovery:} Context-aware therapeutic target identification, enabling reduced off-target effects through tissue-specific network analysis.
    \item \textbf{Biomarker Development:} Tissue-specific interaction signatures as candidates for diagnostic applications.
    \item \textbf{Community Resource:} A publicly accessible, continuously updated interactome database for the systems biology community.
\end{itemize}



\subsection*{Risk Assessment and Mitigation}

\subsection{Technical Risks}
\begin{itemize}
    \item \textbf{LLM Performance Variability:} While LLMs can be sensitive to prompt phrasing, we mitigate this by implementing ensemble approaches with multiple LLM architectures. Furthermore, we will implement an \textbf{Expert-in-the-Loop Validation} protocol: a random subset of 100 generated reasoning traces will be manually reviewed by the PI to calibrate the automated scoring system and detect subtle hallucinations that automated metrics might miss.
    \item \textbf{Scalability Challenges:} The computational cost of querying LLMs for millions of interactions is high. We address this through our optimized high-throughput inference pipeline on the Mac Studio and by transitioning to efficient distilled models for genome-scale inference (WP1 Task 1.3).
\end{itemize}

\paragraph{Biological Risks}
\begin{itemize}
    \item \textbf{Limited Validation Data:} High-quality tissue-specific PPI ground truth is scarce. We mitigate this by using experimentally validated datasets (synaptosome pulldowns for brain, co-abundance networks for other tissues) and orthogonal validation through structural prediction.
    \item \textbf{Negative Set Definition:} Defining true-negative tissue-specific interactions is challenging. We address this by using organ-specific PPIs from orthogonal tissues (e.g., liver metabolic complexes as negatives for brain predictions) rather than relying on arbitrary expression thresholds.
\end{itemize}


%%%%%%%%%%%%%%%%%%%%%%%%%%%%%%%%%%%%%%%%%%%%%%%%%%%%%%%%%%%%%%%%%%%%%%%%%%%%% 
%%%%  TIMELINE  %%%%%%%%%%%%%%%%%%%%%%%%%%%%%%%%%%%%%%%%%%%%%%%%%%%%%%%%%%%%% 
%%%%%%%%%%%%%%%%%%%%%%%%%%%%%%%%%%%%%%%%%%%%%%%%%%%%%%%%%%%%%%%%%%%%%%%%%%%%% 
\let\theparagraph=\oldpara
\paragraph*{Time considerations}
\vspace{-0.5cm}
\begin{figure}[h]
	\centering
  	\resizebox{\textwidth}{!}{\includestandalone[mode=tex]{gantt/gantt}}
	\caption{Workpackage time considerations in months (M).}
	\label{fig:timeline}
\end{figure}

\subsection{Handling of research data}
Data generated during this project will be used for scientific publications in preprint and peer-reviewed journals. All necessary raw data will be deposited in publicly available repositories (e.g., NCBI GEO, SRA) to ensure reproducibility. Source code and analysis pipelines will be open-sourced on \textbf{GitHub}, while the trained Student Models and distilled weights will be made publicly available on \textbf{HuggingFace} to facilitate community adoption. All data management practices will adhere to FAIR principles.

\subsection{Relevance of sex, gender and/or diversity}
The project analyzes human protein interactions across male and female tissues as provided by the Human Protein Atlas, accounting for biological diversity in interactome modeling.

\section{Project- and subject-related list of publications}
\label{sec:bib}
\printbibliography[notcategory=reviewed, notcategory=nonreviewed, notcategory=patents_pending, notcategory=patents, heading=none, env=bibliographyNUM] 	

\backmatter
%%%%%%%%%%%%%%%%%%%%%%%%%%%%%%%%%%%%%%%%%%%%%%%%%%%%%%%%%%%%%%%%%%%%%%%%%%%%%%% 
%%%%  SUPPLEMENT  %%%%%%%%%%%%%%%%%%%%%%%%%%%%%%%%%%%%%%%%%%%%%%%%%%%%%%%%%%%%% 
%%%%%%%%%%%%%%%%%%%%%%%%%%%%%%%%%%%%%%%%%%%%%%%%%%%%%%%%%%%%%%%%%%%%%%%%%%%%%%% 
\section{Supplementary information on the research context}

\subsection{Ethical and/or legal aspects of the project}

\subsubsection{General ethical aspects}
The project uses publicly available, de-identified omics data and does not involve direct human or animal experimentation.

\subsubsection{Descriptions of proposed investigations involving humans, human materials or identifiable data}
None.

\subsubsection{Descriptions of proposed investigations involving experiments on animals}
None.

\subsubsection{Descriptions of projects involving genetic resources from a foreign country}
None.

\subsubsection{Explanations regarding any possible safety-related aspects}
The project focuses on computational modeling of protein interactions and does not involve dual-use research of concern.

\subsubsection{Considerations on aspects of ecological sustainability}
Computational tasks are optimized for high-efficiency hardware (Apple Silicon) to minimize energy consumption compared to traditional server clusters.

\subsection{Employment status information}
Dr. Chung-Wen Lin, Postdoctoral Researcher at Helmholtz Munich.

\textit{Parental leave:} The applicant took parental leave during July 2022 -- January 2023 and November 2024 -- October 2025. During January 2025 -- October 2025, the applicant worked half-time while on partial parental leave.

\subsection{First-time proposal data}
Chung-Wen Lin.

\subsection{Composition of the project group}
The project group consists of the Principal Investigator and one doctoral researcher.

\subsection{Researchers in Germany with whom you have agreed to cooperate}
None.

\subsection{Researchers abroad with whom you have agreed to cooperate}
None.

\subsection{Researchers with whom you have collaborated scientifically within the past three years}
\begin{itemize}
  	\item Prof. Dr. Pascal Falter-Braun (Helmholtz Munich)
\end{itemize}

\subsection{Project-relevant cooperation with commercial enterprises}
None.

\subsection{Project-relevant participation in commercial enterprises}
None.

\subsection{Scientific equipment}
Access to institute HPC resources for structural modeling.

\subsection{Other submissions}
None.

\subsection{Other information}
In submitting a proposal to the DFG, I agree to adhere to the DFG's rules of good scientific practice and FAIR principles.

\paragraph{Use of Generative Models for Content Creation}
In alignment with the DFG's "Guidelines on the Use of Generative Models for Content Creation" (September 2023 and updated 2025), we declare that generative models (LLMs) are used as core research objects within this proposal (e.g., for PPI reasoning). For the preparation of this proposal, LLMs were used for language editing and structural organization; however, all scientific claims, data interpretation, and technical designs remain the sole responsibility of the applicant.

%%%%%%%%%%%%%%%%%%%%%%%%%%%%%%%%%%%%%%%%%%%%%%%%%%%%%%%%%%%%%%%%%%%%%%%%%%%%%%% 
%%%%  REQUESTED MODULES/FUNDS  %%%%%%%%%%%%%%%%%%%%%%%%%%%%%%%%%%%%%%%%%%%%%%%%%%% 
%%%%%%%%%%%%%%%%%%%%%%%%%%%%%%%%%%%%%%%%%%%%%%%%%%%%%%%%%%%%%%%%%%%%%%%%%%%%%%% 
\section{Requested modules/funds}

\subsection{Basic Module}

\subsubsection{Funding for Staff}
We request funding for one doctoral researcher (PhD Student, TV-L 13, 65\%) for 36 months. 

\begin{tabular}{lr}
\textbf{Item} & \textbf{Amount (\euro)} \\
\hline
Doctoral Researcher (TV-L 13, 65\%), 36 months & 152\,100 \\
\hline
\end{tabular}

\subsubsection{Direct Project Costs}

\subsubsubsection{Equipment up to 10\,000\,\euro, Software and Consumables}
None.

\subsubsubsection{Travel Expenses}
We apply for a total of \textbf{10\,000\,\euro} for travel expenses.

\subsubsubsection{Visiting Researchers}
None.

\subsubsubsection{Expenses for Laboratory Animals}
None.

\subsubsubsection{Other Costs}
We request funds for commercial Large Language Model (LLM) API services and subscriptions.

\begin{tabular}{lr}
\textbf{Item} & \textbf{Amount (\euro)} \\
\hline
LLM API Services (Claude, OpenAI) for Benchmarking & 10\,800 \\
\hline
\end{tabular}

\textit{Justification:} While the primary inference will be performed locally on the Mac Studio, access to state-of-the-art commercial models (e.g., Anthropic Claude, GPT-4o) is essential for (1) establishing a "gold standard" reasoning baseline to evaluate our local distilled models, (2) accelerating code development for the web platform, and (3) accessing specialized knowledge bases not yet available in open weights. The estimated cost is based on a monthly usage of approx. 300\,\euro\ for 36 months.

\subsubsubsection{Project-related publication expenses}
We apply for a total of \textbf{1\,500\,\euro} for publication expenses.

\subsubsection{Instrumentation}

\subsubsubsection{Equipment exceeding 10\,000\,\euro}
\textbf{High-Performance Node (512GB RAM):} 15\,000\,\euro.

\subsubsubsection{Major Instrumentation exceeding 50\,000\,\euro}
None.

\subsection{Module Temporary Position for Principal Investigator}
We request a full-time position for the Principal Investigator (Eigene Stelle) for 36 months. 

\begin{tabular}{lr}
\textbf{Item} & \textbf{Amount (\euro)} \\
\hline
Principal Investigator (TV-L 13/14, 100\%), 36 months & 252\,000 \\
\hline
\end{tabular}

\end{document}