\subsection*{Risk Assessment and Mitigation}

\subsection{Technical Risks}
\begin{itemize}
    \item \textbf{LLM Performance Variability:} While LLMs can be sensitive to prompt phrasing, we mitigate this by implementing ensemble approaches with multiple LLM architectures. Furthermore, we will implement an \textbf{Expert-in-the-Loop Validation} protocol: a random subset of 100 generated reasoning traces will be manually reviewed by the PI to calibrate the automated scoring system and detect subtle hallucinations that automated metrics might miss.
    \item \textbf{Scalability Challenges:} The computational cost of querying LLMs for millions of interactions is high. We address this through our optimized high-throughput inference pipeline on the Mac Studio and by transitioning to efficient distilled models for genome-scale inference (WP1 Task 1.3).
\end{itemize}

\paragraph{Biological Risks}
\begin{itemize}
    \item \textbf{Limited Validation Data:} High-quality tissue-specific PPI ground truth is scarce. We mitigate this by using experimentally validated datasets (synaptosome pulldowns for brain, co-abundance networks for other tissues) and orthogonal validation through structural prediction.
    \item \textbf{Negative Set Definition:} Defining true-negative tissue-specific interactions is challenging. We address this by using organ-specific PPIs from orthogonal tissues (e.g., liver metabolic complexes as negatives for brain predictions) rather than relying on arbitrary expression thresholds.
\end{itemize}

