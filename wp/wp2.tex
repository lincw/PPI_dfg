\paragraph{WP2: Structural Validation and Genome-Scale Extension (Year 2)}
\label{wp:2}

\textbf{Objective:} To validate the biological plausibility of LLM-derived interactions using physical structural constraints and to extend predictions to the entire proteome via a Hybrid Neuro-Symbolic approach combining Graph Neural Networks (GNN) with LLM-derived semantic features.

\subparagraph{Task 2.1: Structural Validation via AlphaFold-Multimer (Months 13-18)}
To address the lack of high-throughput experimental validation, we will implement a Physical Gate by leveraging the institute's High-Performance Computing (HPC) resources.
\begin{itemize}
    \item \textbf{Target Selection:} We will implement a tiered validation strategy: (1) For PPIs with existing complex structures in PDB or published interface data, we directly extract pLDDT and ipTM metrics; (2) Novel pairs ($\approx$5,000) will be screened using rapid methods (e.g., ESMFold); (3) A focused subset of $\approx$500 high-priority pairs—including those with ClinVar pathogenic variants at predicted interfaces—will undergo full AlphaFold-Multimer validation. This tiered approach provides robust orthogonal evidence while remaining computationally tractable.
    \item \textbf{Dual-Gating Analysis:} We will correlate the Environmental Probability with the Physical Probability from HPC structural modeling.
\end{itemize}

\subparagraph{Task 2.2: Hybrid Neuro-Symbolic Extension via Graph Neural Networks (Months 19-24)}
While WP1 covers known interactions, we aim to discover novel interactions across the entire proteome. Preliminary results across 49 tissues demonstrate that LLM reasoning captures functional relationships invisible to expression-based methods: interactions ``rescued'' by the LLM (low expression signal, high LLM score) showed 4.3-fold higher pathway validation rates than expression-only predictions. Furthermore, a hybrid approach combining GNN topology with LLM semantics outperformed either method alone in 7 of 10 tested tissues.
\begin{itemize}
    \item \textbf{Heterogeneous Graph Construction:} We will construct tissue-specific Heterogeneous Graphs where nodes represent proteins and edges represent the LLM-derived probabilities. Using a heterogeneous architecture (e.g., Heterogeneous Graph Transformer) allows the model to explicitly account for distinct tissue-specific node features and edge types.
    \item \textbf{Training Strategy:} The GNN will be trained on the Student model's outputs to learn the general, transferable principles of tissue specificity. The Mac Studio transitions during this phase from high-throughput inference (WP1) to model development.
    \item \textbf{Inference:} Rather than exhaustive all-to-all prediction, the trained GNN will focus on inferring tissue-specific probabilities for \textbf{prioritized disease-associated subnetworks} and the \textbf{druggable genome}. This targeted expansion to $\sim$10\,million prioritized pairs reduces false positives inherent in genome-wide screening while maximizing translational impact for drug discovery.
\end{itemize}

\noindent\myTip[1.0\textwidth]{t}{2}{\emph{\vspace*{0.1cm}\textbf{Summary of WP2}\newline
\textbf{Aim:} Validate predictions physically and extend them to the full genome.\newline
\textbf{Method:} AlphaFold structural prediction combined with a Hybrid Neuro-Symbolic GNN that integrates topological and LLM-derived semantic features.\newline
\textbf{Outcome:} A physically-validated, genome-scale atlas of tissue-specific protein interactions with orthogonal validation.}}