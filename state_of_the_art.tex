\subsection*{State of the art: From Statistical Correlation to Biological Reasoning}

The reconstruction of tissue-specific interactomes represents a frontier in systems biology, moving the field from static "parts lists" to context-aware functional models. However, current computational methodologies face a fundamental "reasoning gap" that limits their predictive accuracy and biological interpretability.

\paragraph{Limitations of Co-expression and Statistical Correlation}
Traditionally, tissue-specific protein-protein interactions (PPIs) have been inferred by filtering global interactomes through mRNA co-expression data. While co-expression is a necessary prerequisite for interaction (the law of mass action), it is demonstrably insufficient. A significant fraction of co-expressed proteins never interact due to mismatched subcellular localization, incompatible molecular functions, or low protein-level abundance. Recent studies, notably by \textcite{LamanTrip2025}, have shown that protein co-abundance (derived from mass spectrometry) provides a superior signal compared to mRNA (AUROC 0.80 vs 0.70). However, even co-abundance remains a \emph{proxy} for interaction, failing to capture the structural and biochemical logic that dictates whether two proteins \emph{actually} bind in a specific physiological context.

\paragraph{Critique of Contemporary AI Architectures}
Modern deep learning approaches have attempted to address these limitations but remain constrained by their architectural assumptions and scalability bottlenecks:
\begin{itemize}
    \item \textbf{Graph Neural Networks (SPIDER, PINNACLE):} While SPIDER \cite{SPIDER} utilizes graph attention, it typically treats protein networks as homogeneous graphs, failing to explicitly model the distinct edge types (e.g., regulatory vs. physical) found in different tissues. Similarly, PINNACLE \cite{PINNACLE} generates context-aware embeddings but lacks a mechanism to incorporate unstructured clinical knowledge (e.g., pathogenic variants from ClinVar) into its prediction loop.
    \item \textbf{Structural Modeling Limitations:} Recent studies \cite{LamanTrip2025} have begun to use AlphaFold for validation. However, applying computationally expensive structural modeling to genome-scale networks remains intractable. Current workflows lack a \emph{tiered strategy} to efficiently filter candidates, often resulting in either prohibitively high costs or insufficient coverage.
\end{itemize}

\paragraph{The Emerging Paradigm: Hybrid Neuro-Symbolic Reasoning}
The breakthrough lies in a \emph{Hybrid Neuro-Symbolic} approach. Unlike traditional black-box models, Large Language Models (LLMs) can act as "reasoning engines," integrating semantic evidence—RNA profiles, subcellular localization, and literature-derived disease context—into a unified probabilistic framework. In our preliminary work, this approach achieved an AUROC of 0.86 on experimentally validated brain-specific PPIs, surpassing both co-expression (0.70) and co-abundance (0.80) baselines. By coupling this semantic reasoning with a \textbf{Heterogeneous Graph Neural Network} (which captures topological nuances) and a \textbf{Tiered Structural Validation} strategy (which provides physical ground truth at scale), this project aims to bridge the gap between statistical correlation and functional reality.
